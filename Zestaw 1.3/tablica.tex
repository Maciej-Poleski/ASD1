\documentclass[a4paper,12pt]{article}
%\documentclass[a4paper,12pt]{scrartcl}

\usepackage[utf8x]{inputenc}
\usepackage{polski}

\title{Zadania domowe. Zestaw 1.3}
\author{Maciej Poleski}
\usepackage{amsmath}
\usepackage{amsfonts}
\usepackage{amssymb}
\usepackage{alltt}
\usepackage{listings}

\date{\today}

\pdfinfo{%
  /Title    (Zadania domowe. Zestaw 1.3)
  /Author   (Maciej Poleski)
  /Creator  (Maciej Poleski)
  /Producer (Maciej Poleski)
  /Subject  (MP)
  /Keywords (MP)
}

\begin{document}
\maketitle

\newpage

\section{}
Zakładam że rozpoczyna gracz A. Czyli pytanie brzmi: ile jest pozycji wygrywających dla gracza rozpoczynającego grę. \verb|B| jest funkcją zwracającą wartość wyrażenia logicznego podanego w argumencie (\verb|true| lub \verb|false|).
\begin{alltt}
V <- zbiór wierzchołków
T[V] <- czy gracz zaczynający z wierzchołkiem v wygrywa
            (zostanie obliczone)
a <- 0 (ile jest pozycji wygrywających (dla pierwszego gracza))
for v in V w odwrotnej kolejności topologicznej:
    T[v] <- B(istnieje krawędź z wierzchołka v do wierzchołka w 
                takiego że T[w] = false)
    if T[v] = true:
        a <- a + 1

\end{alltt}
 Kolejność odwrotna topologiczna to kolejność w której "obgryzamy" graf. Czyli zaczynamy od wierzchołków które nie mają krawędzi wyjściowych. Jeżeli jesteśmy w stanie pozostawić drugiemu graczowi stan przegrywający, to nasz stan jest wygrywający. (Poprzez wykonanie posunięcia, które spowoduje powstanie stanu przygrywającego i pozostawienie drugiego gracza w tym stanie). Jeżeli nie jesteśmy w stanie pozostawić drugiemu graczowi stanu przegrywającego to znaczy że albo każdy ruch doprowadzi do stanu wygrywającego (i pozostawimy drugiego gracza w stanie wygrywającym), albo nie mamy możliwości wykonania żadnego ruchu (czyli przegrywamy). Odpowiedź to $\frac{a}{n}$ (taka jest szansa, że wybrane zostanie pole wygrywające dla gracza A, czyli przegrywające dla gracza B). Złożoność głównej pętli jest oczywiście równa $|V|=n$ sprawdzenie czy istnieje krawędź kosztuje nas łącznie $m$. Sortowanie topologiczne (omówione na MP) kosztuje $n+m$. Całość $O(n+m)$.
\section{}
\section{}
\begin{alltt}
V <- zbiór wierzchołków
A[V] <- 1 (ilość zbiorów niezależnych w poddrzewie v 
            takich że v nie należy do tych zbiorów)
B[V] <- 1 (ilość zbiorów niezależnych w poddrzewie v 
            takich że v należy do tych zbiorów)
calc(v):
    for each w in V taki że w jest synem v:
        calc(w)
    for each w in V taki że w jest synem v:
        A[v] <- A[v] * (A[w] + B[w])
        B[v] <- B[v] * A[w]

\end{alltt}
Wywołujemy tę funkcję dla dowolnego wybranego wierzchołka $v$. Odpowiedzią będzie wtedy \verb|A[v] + B[v]|. Obie pętle będą miały łącznie dokładnie $n-1$ obiegów. W efekcie złożoność $O(n)$. Funkcja \verb|calc| jest wykonywana w kolejności postorder. Dla każdego poddrzewa o korzeniu $v$ oblicza ona liczbę zbiorów niezależnych przy założeniu że $v$ do nich należy oraz to samo przy założeniu że nie należy. Jeżeli wierzchołek $v$ należy, to wtedy żadne z jego dzieci nie może należeć. Jeżeli nie należy, to nie ma to znaczenia. Sama ilość tych zbiorów to po prostu ilość wszystkich możliwych kombinacji doboru zbiorów w poddrzewach dzieci wierzchołka $v$ spełniających nasze wymagania (ponieważ mamy do czynienia z drzewem - poddrzewa nie są połączone żadnymi krawędziami, trzeba zająć się jedynie korzeniem).


\end{document}
