\documentclass[a4paper,12pt]{article}
%\documentclass[a4paper,12pt]{scrartcl}

\usepackage[utf8x]{inputenc}
\usepackage{polski}

\title{Zadania domowe. Zestaw 2.2}
\author{Maciej Poleski}
\usepackage{amsmath}
\usepackage{amsfonts}
\usepackage{amssymb}
\usepackage{alltt}
\usepackage{listings}

\date{\today}

\pdfinfo{%
  /Title    (Zadania domowe. Zestaw 2.2)
  /Author   (Maciej Poleski)
  /Creator  (Maciej Poleski)
  /Producer (Maciej Poleski)
  /Subject  (ASD1)
  /Keywords (ASD1)
}

\begin{document}
\maketitle

\newpage

\section{}

\section{}
Posortować według niemalejących czasów obsługi i obsługiwać po kolei. Weźmy dowolne rozwiązanie optymalne (kolejność obsługi klientów). Zmodyfikujmy to rozwiązanie przenosząc najbardziej czasochłonnego klienta na koniec kolejki. Koszt obsługi ostatniego klienta nie zależy od kolejności obsługi - liczy się tylko czas zakończenia, który bez względu na kolejność jest taki sam (dodawanie jest przemienne). Mamy taką sytuacje (w rozwiązaniu optymalnym):
\begin{verbatim}
------X-------Y
\end{verbatim}
\verb|Y| jest ostatni, \verb|X| jest najbardziej kosztowny. Jeżeli zamienimy ich miejscami, to wszystkie koszta przed \verb|X| pozostaną takie same, koszt ostatniego klienta pozostanie taki sam, a koszta klientów od \verb|X| do \verb|Y| nie wzrosną ponieważ \verb|Y|$\leq$\verb|X|.
\begin{alltt}
 A[0..n-1] <- tablica z kosztami obsługi wszystkich klientów
 sort(A)
 result <- 0
 prefix <- 0
 for i <- to n):
    prefix <- prefix + A[i]
    result <- result + prefix

\end{alltt}
Odpowiedź jest w zmiennej \verb|result|.
\section{}
Zauważmy że trzecie ograniczenie jest nadmiarowe. Wszystkie elementy zbioru pustego zostały wybrane (zawsze). Więc trzecie ograniczenie jest szczególnym przypadkiem pierwszego. Pozostałe ograniczenia możemy utożsamiać z formułami logicznymi postaci $$x_i \wedge x_{i+1} \wedge \ldots \wedge x_{i+k} \rightarrow x_l$$. Wtedy warunek drugi będzie realizowany formułą dla której $x_l = \bot$ i będziemy szukać odpowiedzi na pytanie czy istnieje takie wartościowanie zmiennych $x_1 \ldots x_n$ dla którego każda formuła ma wartość 1.

Oto jak tego dokonamy: Będziemy przechowywać kolekcje wszystkich ograniczeń oraz informację pozwalającą dla zadanej zmiennej natychmiast (w czasie stałym) uzyskać formuły w których występuje. Gdy określimy że wartość pewnej zmiennej wynosi 1 - usuniemy ją ze wszystkich ograniczeń w których występuje jako poprzednik implikacji (jako element neutralny logicznej koniunkcji). Gdy zachodzi (pusto) spełnienie poprzednika implikacji wiemy że następnik musi mieć wartość 1. Jeżeli jego wartość jeszcze nie jest znana - właśnie ją poznaliśmy. Jeżeli wynosi 0 - mamy sprzeczność (rozwiązanie nie istnieje, ponieważ wszystkie zmienne których wartość określiliśmy do tej pory są określone jednoznacznie). Wykonujemy tą operację do oporu. Jeżeli udało się - wartość wszystkich do tej pory nie ustalonych zmiennych możemy uznać za 0 - dzięki temu wszystkie implikacje będą spełnione. (z fałszu wynika wszystko...).
\begin{alltt}
 queue <- kolejka zmiennych których wartość określiliśmy na 1
 --przeglądnij wszystkie ograniczenia i umieść w kolejce
 implikacje których poprzednik jest (pusto) spełniony--
 while queue is not empty:
    x <- queue.dequeue()
    usuń wszystkie wystąpienia x ze zbioru ograniczeń
    dodaj do kolejki wszystkie zmienne które w efekcie stały
       się następnikiem implikacji z pustym poprzednikiem
    if istnieje ograniczenie którego poprzednik jest spełniony
       a następnik fałszywy:
       ROZWIĄZANIE NIE ISTNIEJE
 ROZWIĄZANIE ISTNIEJE
\end{alltt}
A teraz jak zaimplementować wnętrze pętli, aby uzyskać oczekiwaną złożoność. Ograniczenia możemy identyfikować za pomocą identyfikatora (liczby porządkowej). Wystarczy, że dla każdego ograniczenia będziemy pamiętać ile ma poprzedników i jaki jest jego następnik (i czy jest on $\bot$). Oprócz tego dla każdej zmiennej (identyfikowanej np liczbą porządkową) zapamiętujemy w których ograniczeniach ona występuje (jako poprzednik implikacji). Dzięki temu usuwamy $x$ w czasie proporcjonalnym do ilości jego wystąpień w ograniczeniach (zmniejszając zapamiętaną liczbą poprzedników o 1). Jeżeli jakieś ograniczenie w efekcie osiągnęło 0 poprzedników sprawdzamy jaki ma następnik i dodajemy go do kolejki (jeżeli jest on fałszywy - ROZWIĄZANIE NIE ISTNIEJE). Taki preprocesing trwa proporcjonalnie do długości formuł niemal bez względu na to jak są one wyrażone na wejściu, a po jego wykonaniu już nie korzystamy z danych wejściowych. W efekcie cały algorytm działa w czasie proporcjonalnym do długości formuł.
\section{}
Zasadniczo jest to niemalże problem optymalnego nawiasowania. Dla każdego infiksu wyznaczymy jakie wartości może on przyjąć po nawiasowaniu (na pewno będzie co najmniej jedna wartość - jakiś wynik musi powstać). 
\begin{alltt}
 T[0..n-1][0..n-1] <- T[i][j] jest zbiorem możliwych do osiągnięcia
                                wyników infiksu x_i ... x_j
 T[i][i] <- \{x_i\}
 
 stage1(i,j):
    if T[i][j]!=\(\varnothing\)
        return T[i][j] -- już policzone
    for k <- i to j):
        for each x in stage1(i,k):
            for each y in stage1(k+1,j):
                T[i][j] <- T[i][j] \(\cup\) \{x\(\circ\)y\}

 stage2(i,j,z):
    if i = j:
        print z
        return
    print '('
    for k <- i to j):
        for each x in stage1(i,k):
            for each y in stage1(k+1,j):
                if x\(\circ\)y = z:
                    stage2(i,k,x)
                    print ')\(\circ\)('
                    stage2(k+1,j,y)
                    break wszystkie pętle
                    
    print ')'
    
 main:
    stage1(0,n-1)
    if a \(\in\) T[0][n-1]:
        stage2(0,n-1,a)
    else:
        NIE DA SIĘ
\end{alltt}
Uruchomić \verb|main|. Złożoność jest taka sama jak w optymalnym nawiasowaniu (wykład).

\end{document}
