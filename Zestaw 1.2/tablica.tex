\documentclass[a4paper,12pt]{article}
%\documentclass[a4paper,12pt]{scrartcl}

\usepackage[utf8x]{inputenc}
\usepackage{polski}

\title{Zadania domowe. Zestaw 1.2}
\author{Maciej Poleski}
\usepackage{amsmath}
\usepackage{amsfonts}
\usepackage{amssymb}
\usepackage{alltt}
\usepackage{listings}

\date{\today}

\pdfinfo{%
  /Title    (Zadania domowe. Zestaw 1.2)
  /Author   (Maciej Poleski)
  /Creator  (Maciej Poleski)
  /Producer (Maciej Poleski)
  /Subject  (MP)
  /Keywords (MP)
}

\begin{document}
\maketitle

\newpage

\section{}

\section{}
\begin{alltt}
 T[1..n] <- wartości sztabek. T[1] to wartość sztabki na samym dnie stosu
 F[0] <- 0
 S[0..M] <- 0
 for i <- 1 to min(M,n)]:
    F[i] <- F[i-1]+T[i]
 for i <- M+1 to n]:
    F[i] <- T[i] + S[i-1]
    S[i] <- F[i-1]
    for j <- 2 to M]:
        if S[i-j] + T[i] + T[i-1] + ... + T[i-j+1] > F[i]:
            F[i] <- S[i-j] + T[i] + T[i-1] + ... + T[i-j+1]
            S[i] <- F[i-j]
\end{alltt}
 W tablicy \verb|F[0..n]| znajduje się odpowiedź na pytanie jaka jest łączna wartość sztabek zabranych ze stosu 0..n (ostatnie n sztabek) przez pierwszego gracza, jeżeli obaj grają optymalnie. W \verb|S[0..n]| jest odpowiedź dla drugiego gracza. Algorytm określa optymalny wynik dla danego stosu korzystając z informacji o optymalnych wynikach dla mniejszych stosów (po prostu wybiera najlepsze zagranie z pośród wszystkich legalnych). Ponieważ $M$ jest stałą, złożoność całego algorytmu jest liniowa względem n. Odpowiedź jest zapisana w \verb|F[n]|.
\section{}
Optymalna ścieżka do pozycji (i,j) może powstać na dokładnie jeden z dokładnie dwóch sposobów: przechodzimy na tę pozycję z pozycji (i-1,j) albo (i,j-1). Znając więc optymalne rozwiązanie dla pola powyżej i z lewej możemy z łatwością wskazać optymalne rozwiązanie dla danego pola.
\begin{alltt}
T[n][n] <- plansza zgodnie z opisem w zadaniu
\(\forall{i<0} \forall{j<0}\) T[i][j] <- 0
for i <- 0 to n):
    for j <- 0 to n):
        T[i][j] <- T[i][j] + max{T[i-1][j],T[i][j-1]}

\end{alltt}
Odpowiedź jest zapisana w \verb|T[n-1][n-1]|. Złożoność jest raczej oczywista.


\end{document}
