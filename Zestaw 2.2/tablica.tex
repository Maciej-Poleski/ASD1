\documentclass[a4paper,12pt]{article}
%\documentclass[a4paper,12pt]{scrartcl}

\usepackage[utf8x]{inputenc}
\usepackage{polski}

\title{Zadania domowe. Zestaw 3.1}
\author{Maciej Poleski}
\usepackage{amsmath}
\usepackage{amsfonts}
\usepackage{amssymb}
\usepackage{alltt}
\usepackage{listings}

\date{\today}

\pdfinfo{%
  /Title    (Zadania domowe. Zestaw 3.1)
  /Author   (Maciej Poleski)
  /Creator  (Maciej Poleski)
  /Producer (Maciej Poleski)
  /Subject  (ASD1)
  /Keywords (ASD1)
}

\begin{document}
\maketitle

\newpage

\section{}
Proces nie zakończy się jeżeli dojdziemy do cyklu. Czyli naszym zadaniem jest stwierdzenie z których wierzchołków nie da się dotrzeć do cyklu. Generalnie zachodzi zależność, że jeżeli z wierzchołka x da się dotrzeć do wierzchołka y z którego da się dotrzeć do cyklu, to z wierzchołka x da się dotrzeć do cyklu. Zastosuję standardowy DFS szukający cyklu + tą przechodniość.
\begin{alltt}
 C[v] <- czy z wierzchołka v da się dotrzeć do cyklu
 state[0..n] <- READY
 dfs(v):
    if state[v] = BUSY:
        return CYKL
    else if state[v] = DONE:
        return C[v]
    state[v] <- BUSY
    for each (v,u) \(\in\) E:
        if dfs(u) = CYKL:
            state[v] <- DONE
            C[v] <- CYKL
            return C[v]
    state[v] <- DONE
    C[v] <- OK
    return C[v]
    
 for each v \(\in\) V:
    if dfs(v) = OK:
        print v
    
\end{alltt}
Koszt algorytmu to jednokrotne przejście po całym grafie. $O(n+m)$

\section{}
Wewnątrz każdej silnie spójnej składowej można podróżować bez ograniczeń (z każdego wierzchołka możemy dotrzeć do każdego innego). Skoro tak to problem sprowadza się do odbycia podróży z silnie spójnej składowej wierzchołka x do składowej wierzchołka y (lub odwrotnie). Ale teraz już w DAG-u. Zadanie sprowadza się w takim razie do rozstrzygnięcia czy DAG jest porządkiem liniowym. (Jeżeli jest - oczywiście graf jest średnio-spójny, jeżeli nie - istnieją dwa wierzchołki nieporównywalne - czyli nie jest średnio-spójny).
\begin{alltt}
 Wyznacz podział na silnie spójne składowe
 S[s] <- \(\varnothing\) - składowa do której istnieje krawędź z składowej s
 U[s] <- \(\varnothing\) - składowa z której istnieje krawędź do składowej s
 for each (v,u) \(\in\) E:
    if v jest w innej składowej niż u:
        if S[id-składowej(v)] != \(\varnothing\) and S[id-składowej(v)] != id-składowej(u):
            GRAF NIE JEST ŚREDNIO-SPÓJNY
        S[id-składowej(v)] <- id-składowej(u)
        if U[id-składowej(u)] != \(\varnothing\) and U[id-składowej(u)] != id-składowej(v):
            GRAF NIE JEST ŚREDNIO-SPÓJNY
        U[id-składowej(u)] <- id-składowej(v)
 CS <- 0
 CU <- 0
 for each s będącego id jakiejś składowej:
    if S[s] = \(\varnothing\):
        CS <- CS + 1
    if U[s] = \(\varnothing\):
        CU <- CU + 1
    
 if CU != 1 or CS != 1:
    GRAF NIE JEST ŚREDNIO-SPÓJNY
 GRAF JEST ŚREDNIO-SPÓJNY
\end{alltt}
Najpierw wyznaczam podział na silnie spójne składowe. Przykładowe rozwiązanie tego problemu można znaleźć w zadaniach Q i R. Następnie sprawdzam czy porządek jest liniowy (czyli czy dla każdej składowej istnieje dokładnie jeden poprzednik i dokładnie jeden następnik z wyjątkiem dokładnie jednej składowej która nie ma poprzednika i dokładnie jednej która nie ma następnika). Podział na SSS $O(n+m)$, pętla po krawędziach $O(m)$, pętla po składowych $O(n)$. Całość $O(n+m)$.

\section{}

\section{}
Przedstawię rozwiązanie w złożoności $O(n^2+m)$. Problem można sprowadzić do 2-SAT-u. Dla każdej krawędzi \{u,v\} mamy formułę $(u\vee{v})$ mówiącą że co najmniej jedno z miast u lub v musi być stolicą. Następnie dla każdej pary miast \{u,v\} z tego samego województwa mamy formułę $\neg{u}\vee{\neg{v}}$ mówiącą że co najwyżej jedno z miast u lub v może być stolicą. W ten sposób gwarantujemy że w województwie nie istnieje para miast które są stolicą, czyli że istnieje co najwyżej jedna stolica (jeżeli uzyskamy rozwiązanie instancji w którym nie istnieje stolica, to możemy bezpiecznie dodać jedną dowolną - siłą rzeczy nie mamy szans uszkodzić w ten sposób rozwiązania). Mamy w ten sposób $n^2+m$ formuł i $n$ zmiennych. Możemy znaleźć rozwiązanie tej instancji w czasie $O(n+n^2+m)=O(n^2+m)$. Zasadniczo chcemy wiedzieć jedynie czy rozwiązanie istnieje. Możemy wykorzystać zadanie R które rozwiązuje ten problem. (Szkoda przeklejać kodu - to jest dokładnie to samo).

\end{document}
